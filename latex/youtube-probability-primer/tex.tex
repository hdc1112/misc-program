% https://latexbase.com/

\documentclass[12pt]{article}
\usepackage{amsmath}
\usepackage{graphicx}
\usepackage{amssymb}
\usepackage{hyperref}
\usepackage[latin1]{inputenc}

\title{Notes on Youtube's probability primer}
\date{\today}

\newcommand{\A}{\mathcal{A}}
\newcommand{\B}{\mathcal{B}}
\newcommand{\C}{\mathcal{C}}
\newcommand{\T}{\mathcal{T}}

\begin{document}
\maketitle

\section{Definition of $\sigma$-algebra}

Given a set $\Omega$, a $\sigma$-algebra $\A$ is a non-empty subset of $2^\Omega$ such that:

\begin{enumerate}
    \item closed under complement, $E\in \A \implies E^c \in \A$ 

    \item closed under countable unions, $E_1\in \A, E_2\in \A \implies E_1\cup E_2\in \A$
\end{enumerate}

\par \noindent

\textbf{Theorem 1} $\Omega\in \A$

Proof: since $\A$ is non-empty, there must exist one element $E\in \A$. Because $\A$ is closed
under complement, so $E^c\in \A$. Because of $\A$ is closed under countable unions, so
$E \cup E^c \in \A$. Hence $\Omega\in A$.

\textbf{Theorem 2} $\phi \in \A$

Proof: because $\Omega \in \A$ and $\A$ is closed under complement, so $\Omega^c=\phi\in \A$.

\textbf{Theorem 3} $\A$ is closed under countable intersections.

Proof: $\bigcap\limits_{i=1}^{\infty} E_i = $(\bigcup\limits_{i=1}^{\infty} E_i^c)^c$
From this equation we can see that $\A$ is closed under intersections.

\section{Definition of generated $\sigma$-algebra}

Given a $\C\in 2^\Omega$, the generated $\sigma$-algebra $\C$, or $\sigma(\C)$, is the smallest
$\sigma$-algebra containing $\C$.
Or, $\sigma(\C)=\bigcap\limits_{\A\ is\ \sigma-algebra\ on\ \Omega,\ \C\subseteq \A} \A$.

\textbf{Theorem 4} $\sigma(\C)$ always exists

Proof: $2^\Omega$ is a $\sigma$-algebra, and any intersections of $\sigma$-algebra is also
$\sigma$-algebra (needed by the mathematical ``smallest" definition). (These two can be easily proved.)

Example of $\sigma$-algebra:
\begin{enumerate}
    \item $\A=\{\phi, \Omega\}$
    \item $\A=\{\phi,E, E^c, \Omega\}$
    \item if $\Omega = \mathbb{R}$, the Borel $\sigma$-algebra $\B=\sigma(\T)$ where
        $\T=\{open\ sets\ of\ \mathbb{R}\}$.
\end{enumerate}

\section{Definition of measure}

A measure $\mu$ on $\Omega$ with $\sigma$-algebra $\A$ is a function
$\mu: \A \rightarrow [0,\infty]$ such that:

\begin{enumerate}
    \item $\mu(\phi) = 0$
    \item ``countable additivity" $\mu(\bigcup\limits_{i=1}^{\infty} E_i)=\sum\limits_{i=1}^{\infty} \mu(E_i)$ for any pairwise disjoint sets $E_1, E_2, ... \in \A$ 
\end{enumerate}

\section{Definition of probability measure}

A probability measure is a measure $P$ such that $P(\Omega) = 1$. (Notice that $\Omega\in \A$).

\end{document}

